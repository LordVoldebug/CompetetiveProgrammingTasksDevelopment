\begin{problem}{На брезентовом поле...}{стандартный ввод}{стандартный вывод}{4 секунды}{256 мегабайт}

Фермер Витя сажает алюминиевые огурцы и оловянные помидоры (на брезентовом поле, а как же иначе). Брезентовое поле представляет собой матрицу размерами $n$ на $m$ ($n$ строк, $m$ столбцов). К сожалению, у брезентового поля есть одна неприятная особенность: ячейки в этой матрице бывают плодородными и неплодорными. Если посадить металлический овощ в неплодородную ячейку поля, то урожай не взойдёт, причём целиком. Также, нельзя посадить два овоща в одну ячейку поля.

Неприятная особенность есть и у металлических овощей: для каждого типа овощей, урожай взойдёт если и только если область, где растёт овощь данного типо на поле представляет собой прямоугольник со сторонами, паралельными сторонам поля. Другими словами, Витя может выбрать в качестве места для посева не более двух прямоугольных плодородный областей поля.

Витя хочет получить как можно больший урожай со своего поля. Помогите ему! Напишите программу, которая по описанию брезентового поля выдаёт какой максимальный урожай сможет получить Витя.

\InputFile
В первой строке входных данных содержаться два числа $n$ и $m$ - размеры поля, $n * m \le 10 ^ 6$.
В следующих $n$ строках следуют по $m$ символов $0$, $1$ - описание поля. При этом, $1$ соответсвует плодородной ячейке поля, а $0$ - неплодородной.

\OutputFile
Выведите единственное число - максимальный урожай, который сможет получить Витя со своего поля.

\Example

\begin{example}
\exmpfile{example.01}{example.01.a}%
\end{example}

\end{problem}

