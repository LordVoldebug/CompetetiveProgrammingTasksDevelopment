В первой строке входных данных дано число $n$ ($1 \le n \le 5 * 10 ^ 5$)  - количество ягод у Васи.
В последующих $n$ строках описываются параметры каждой ягоды. 
В $i + 1$ строке дано два числа $a_i$ и $b_i$  ($0 \le a_i, b_i \le 10 ^ 9$) - то, насколько поднимется улитка за день после того, как съест $i$ ягоду и насколько опуститься за ночь.