Домашний питомец мальчика Васи - улитка Петя. Петя обитает на бесконечном в обе стороны столбе, представляющим собой числовую прямую. Изначально Петя находится в точке 0.

Вася кормит Петю ягодами. У него есть $n$ различных ягод, каждая в единственном экземпляре.
Вася знает, что если в $i$ день он даст Пете $i$ ягоду, то поев и набравшись сил, за остаток дня Петя поднимется на $a_i$ единиц ввверх по столбу, но при этом за ночь, потяжелев, съедет на $b_i$ единиц вниз.

Пете стало интересно, а как оно там, наверху, и Вася взялся ему в этом помочь. Ближайшие $n$ дней он будет кормить Петю ягодами из своего запаса таким образом, чтобы максимальная высота, на которой побывал Петя за эти $n$ дней была максимальной. К сожалению, Вася не умеет программировать, поэтому он попросил вас о помощи. Найдите, максимальную высоту, на которой Петя сможет побывать за эти $n$ дней и каким образом Вася должен его кормить, чтобы Петя смог её достичь!