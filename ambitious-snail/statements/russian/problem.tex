\begin{problem}{Амбициозная улитка}{стандартный ввод}{стандартный вывод}{1 секунда}{256 мегабайт}

Домашний питомец мальчика Васи - улитка Петя. Петя обитает на бесконечном в обе стороны столбе, представляющим собой числовую прямую. Изначально Петя находится в точке 0.

Вася кормит Петю ягодами. У него есть $n$ различных ягод, каждая в единственном экземпляре.
Вася знает, что если в $i$ день он даст Пете $i$ ягоду, то поев и набравшись сил, за остаток дня Петя поднимется на $a_i$ единиц ввверх по столбу, но при этом за ночь, потяжелев, съедет на $b_i$ единиц вниз.

Пете стало интересно, а как оно там, наверху, и Вася взялся ему в этом помочь. Ближайшие $n$ дней он будет кормить Петю ягодами из своего запаса таким образом, чтобы максимальная высота, на которой побывал Петя за эти $n$ дней была максимальной. К сожалению, Вася не умеет программировать, поэтому он попросил вас о помощи. Найдите, максимальную высоту, на которой Петя сможет побывать за эти $n$ дней и каким образом Вася должен его кормить, чтобы Петя смог её достичь!

\InputFile
В первой строке входных данных дано число $n$ ($1 \le n \le 5 * 10 ^ 5$)  - количество ягод у Васи.
В последующих $n$ строках описываются параметры каждой ягоды. 
В $i + 1$ строке дано два числа $a_i$ и $b_i$  ($0 \le a_i, b_i \le 10 ^ 9$) - то, насколько поднимется улитка за день после того, как съест $i$ ягоду и насколько опуститься за ночь.

\OutputFile
В первой строке выходных данных выведите единственное число - максимальную высоту, которую сможет достичь Петя, если Вася будет его кормить оптимальным образом.
В следующей строке выведите $n$ различных целых чисел от $1$ до $n$ - порядок, в котором Вася должен кормить Петю ($i$ число в строке соответствует номеру ягоды, которую Вася должен дать Пете в $i$ день чтобы Петя смог достичь максимальной высоты).

\Example

\begin{example}
\exmpfile{example.01}{example.01.a}%
\end{example}

\end{problem}

